\chapter{MHD spectroscopy of a solar atmosphere} \label{ch: solar_atmosphere}

\begin{chapterquote}[Jean-Luc Picard][Star Trek TNG: `Attached']
  There is a way out of every box, a solution to every puzzle; it's just a matter of finding it.
\end{chapterquote}

\usespublishedwork{
  Most of this Chapter was published in ``Magnetohydrodynamic Spectroscopy of a Non-Adiabatic Solar Atmosphere'', 2021, Solar Physics, 296, 9 \citep{claes2021}. N. Claes performed the simulations, did the analysis, and wrote the manuscript; R. Keppens contributed to the revision of the paper.
}

In this Chapter we perform a detailed, high-resolution spectroscopic study of the solar atmosphere, using the newly developed {\legolas} code to calculate the full spectrum with corresponding eigenfunctions of equilibrium configurations that are based on fully realistic solar atmosphere models. Physical effects include gravity, optically thin radiative losses, and thermal conduction. We will give special attention to thermal instabilities, known to be responsible for the formation of solar prominences, and present a new outlook on the thermal and slow continua and how they behave in different chromospheric and coronal regions. We will show that thermal instabilities are unavoidable in the solar atmosphere models used in this Chapters, and that there exist entire regions where both the thermal, slow, and fast modes all have unstable wave mode solutions. Special regions are encountered where the slow and thermal continua become purely imaginary, merging on the imaginary axis. These studies mark the first time a detailed eigenspectrum analysis has been done on a self-consistent solar atmosphere model; all spectra discussed in this Chapter illustrate clearly that thermal instabilities (both discrete and continuum modes) and magnetothermal overstable propagating modes are ubiquitous throughout the solar atmosphere, and may well be responsible for much of the observed fine-structuring and multi-thermal dynamics.

\section{Introduction}
The solar atmosphere shows a myriad of thermodynamically fascinating features, from solar prominences to coronal rain, over plumes, jets, etc. Extensive research on all these features has been done over the past decades, and currently nonlinear MHD simulations are starting to resolve aspects at resolutions rivalling upcoming and future solar telescopes.
A fully self-consistent solar flare model achieving resolutions of 50 km was recently shown in \citet{ruan2020}, while \citet{jenkins2021} focused more on the physical processes driving prominence formation, increasing the resolution even further down to 5 km. In realistic solar atmospheric evolutions many linear waves and instabilities can be at play and interact with one another, implying that modern nonlinear simulations can benefit greatly from the full knowledge of all linear instabilities and eigenoscillations of a given configuration. Some early attempts were made by for example \citet{nye1976}, who discussed an analytical model of the solar atmosphere based on an isothermal slab, where certain mode types of the magnetised atmosphere could be computed analytically in ideal MHD. \citet{vanderlinden1991} did some pioneering work in linear MHD including non-adiabatic effects, hinting at the role of magnetothermal modes in prominence fine-structuring, but since then further research into this topic has stalled.

While fully nonlinear, quasi-realistic numerical simulations can accurately model a plethora of physical phenomena, one can not underestimate the importance of linear MHD. Looking at the saturation of a single unstable mode in the context of prominence formation is one thing, but a \emph{detailed} knowledge of which particular mode among a myriad of unstable eigenmodes is responsible for the evolution can only be obtained through a detailed and careful spectroscopic analysis of a given state. As mentioned before, this has been realised by \citet{demaerel2016}, where they showed -- in ideal MHD, including self-gravity -- that spectral theory actually governs the stability of single-fluid evolving time-dependent plasmas, at every point during their nonlinear evolution. Knowing all MHD modes of (possibly coupled mode types) in a magnetised solar atmosphere, and how they modify as a result of including relevant non-adiabatic effects, is thus a clear necessity in studying solar prominences and their intriguing fine structure. In this Chapter we set out to do exactly this for a realistic solar atmosphere model, in which we include all relevant physical effects important for thermal instability and prominence formation -- that is, external gravity, radiative cooling, and anisotropic thermal conduction. To date there exists no detailed MHD spectroscopic treatment of such a self-consistent model.

There are in fact a wide range of existing interesting studies regarding thermal instabilities and waves in non-adiabatic solar atmosphere-like settings. \citet{zavershinskii2019} demonstrated the formation of slow magneto-acoustic wave trains due to linear dispersion associated with heating-cooling imbalance, while recently \citet{duckenfield2021} looked at slow-mode damping rates in solar coronal conditions under the influence of thermal imbalance and showed that these may be intricately linked. \citet{ledentsov2021} on the other hand looked at thermal instability in preflare current layers including viscosity and the effect that a longitudinal magnetic field has on spatial stabilisation. Studies like these are quite limited however, in the sense that they have to drastically reduce their mathematical models to either homogeneous background conditions in 2D or 3D setups, or to 1D treatments along magnetic field lines in order to be able to derive a dispersion relation or to make the problem analytically tractable.

Armed with our new {\legolas} code we are now able to do a complete eigenspectrum analysis of a fully realistic solar atmosphere model, where we adopt the widely used semi-empirical temperature profile proposed by \citet{avrett2008}, thereby including all aforementioned physical effects. We pay special attention to both the thermal and slow continua, both of which play a very important role in the (in)stability of a given equilibrium configuration. The intricate structure of the thermal, slow, and Alfv\'en continua, and the way that the many discrete modes organise themselves in (coupled) thermal, slow, Alfv\'en and fast wave sequences, is demonstrated here for the first time, giving us a linear ``preview'' of how nonlinear simulations should develop as a result of interacting instabilities. The main advantage of using {\legolas} here is that we are not bound by the same limitations as previous works, in the sense that we do not need any dispersion relation or continuous background profiles in order to calculate growth rates, meaning we can go far beyond that which has already been done in existing literature in our treatment of the solar atmosphere.

In Section \ref{sec: analytical_work} we first revisit the analytical work done by \citet{nye1976} and complement this by looking at non-parallel propagation and the inclusion of non-adiabatic effects, something that was impossible in the original analytical treatment. This will serve as a stepping stone to Section \ref{sec: solar_atmosphere}, where we finally look at a fully realistic solar atmosphere model and perform a detailed spectroscopic analysis with special attention to thermal instabilities. Herein we demonstrate that thermal instability is ubiquitously present in both the solar chromosphere and corona, and provide a fresh outlook on mode behaviour in these regions. In the solar chromosphere, we will show that it becomes possible for the slow continua to merge with the imaginary axis, indicating ``upwards'' and ``downwards'' branches that in turn may become unstable if the conditions are right.


\section{Revisiting analytical models} \label{sec: analytical_work}
\subsection{Purely parallel propagation} \label{ss: parallel_propagation}
\subsection{Effects of non-parallel propagation} \label{ss: non_parallel_propagation}
\subsection{Inclusion of non-adiabatic effects} \label{ss: non_adiabatic_effects}

\section{A realistic solar atmosphere model} \label{sec: solar_atmosphere}
\subsection{The stratified, magnetised atmosphere} \label{ss: atmosphere_model}
\subsection{Non-adiabatic continua} \label{ss: non_adiabatic_continua}
\subsection{MHD spectra for the solar corona} \label{ss: corona_spectra}
\subsection{MHD spectra for the solar chromosphere} \label{ss: chromosphere_spectra}

\section{Discussion}



\cleardoublepage
