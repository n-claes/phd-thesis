\chapter{MHD spectroscopy of a solar atmosphere} \label{ch: solar_atmosphere}

\begin{chapterquote}[Jean-Luc Picard][Star Trek TNG: `Attached']
  There is a way out of every box, a solution to every puzzle; it's just a matter of finding it.
\end{chapterquote}

\usespublishedwork{
  Most of this Chapter was published in ``Magnetohydrodynamic Spectroscopy of a Non-Adiabatic Solar Atmosphere'', 2021, Solar Physics, 296, 9 \citep{claes2021}. N. Claes performed the simulations, did the analysis, and wrote the manuscript; R. Keppens contributed to the revision of the paper.
}

In this Chapter we perform a detailed, high-resolution spectroscopic study of the solar atmosphere, using the newly developed {\legolas} code to calculate the full spectrum with corresponding eigenfunctions of equilibrium configurations that are based on fully realistic solar atmosphere models. Physical effects include gravity, optically thin radiative losses, and thermal conduction. We will give special attention to thermal instabilities, known to be responsible for the formation of solar prominences, and present a new outlook on the thermal and slow continua and how they behave in different chromospheric and coronal regions. We will show that thermal instabilities are unavoidable in the solar atmosphere models used in this Chapters, and that there exist entire regions where both the thermal, slow, and fast modes all have unstable wave mode solutions. Special regions are encountered where the slow and thermal continua become purely imaginary, merging on the imaginary axis. These studies mark the first time a detailed eigenspectrum analysis has been done on a self-consistent solar atmosphere model; all spectra discussed in this Chapter illustrate clearly that thermal instabilities (both discrete and continuum modes) and magnetothermal overstable propagating modes are ubiquitous throughout the solar atmosphere, and may well be responsible for much of the observed fine-structuring and multi-thermal dynamics.

\section{Introduction}


\cleardoublepage
