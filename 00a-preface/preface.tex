\chapter*{Preface} \label{ch: preface}
\addcontentsline{toc}{chapter}{\nameref{ch: preface}}

To begin this thesis with a quote from Star Trek Enterprise's theme song: ``\emph{It's been a long road, getting from there to here}''. And indeed it has. Four years is quite a long time, yet they flew by in an instant and I can still remember my first ``official'' day in the office as if it was yesterday. I would like to take this first part of my thesis and use it to acknowledge everyone who made these last four years unique and enjoyable. I can already say that I probably failed to do this, as including \emph{everyone} would make this Preface at least as long as the first three chapters of my thesis. Nevertheless, I'm taking the time here to properly thank the people that contributed to the remarkable experience that was these last four years.

First and foremost, my sincerest gratitude to my promotor, Rony Keppens. Thank you for giving me this opportunity to do a PhD in your group. I can genuinely say that you are one of the best mentors a starting PhD student can wish for. I could always count on you for a quick chat or to discuss one of my (numerous) questions when I was (yet again) stuck on another aspect of spectroscopy. Every piece of guidance and feedback you gave me over these past few years played a huge part in where I am today, and pushed me to further improve my work and gain new knowledge. I honestly could not have wished for a better supervisor during my PhD.

A special word of thanks to my fellow PhD students, without you these past few years would have turned out entirely different. I really enjoyed our morning chats over a cup of coffee, though our actual ``office days'' were limited for quite a while. A big shout-out to Jordi for all the fruitful science discussions we had when piecing Legolas together, you kept me sane multiple times when going down the rabbit hole of spectroscopy and finite element analysis. Without your help we wouldn't have gotten where we are today. My office buddies Veronika and Brecht, who where always up for a chat after a joyful ``good morning'', and you too Joris, always happy to pop over from the office across to join us. Julia, for all the entertaining discussions during lunch, too bad we were forced to work from home right after I moved to the same office. I did put your nerf gun to good use, though. Last but not least, Nicolas, Hanne and Dion, even though you all started in my final year it was a joy having you around to join our coffee breaks.

I would also like to mention a few postdocs: Ileyk and Jannis, thanks for always being available whenever I needed help understanding coding aspects. Of course Jack, you were an absolute joy to have around. You brought the fun with you whenever you joined for coffee, lunch or drinks, lecturing us on the difference between cookies, biscuits and crumpets.
I enjoyed our many discussions, hereby my apologies that you had to sometimes hear me rant on about science things I was struggling with. Also Jean-Baptiste, the lengthy chats we had were invaluable to get a new perspective on things.

Special thanks to Marcel Goossens and Hans Goedbloed, the many helpful discussions we had on spectroscopy and Legolas sparked a lot of new ideas and interesting questions, though I only got around tackling a few of them. Thanks to Peng-Fei Chen, for giving me the opportunity to spend three months in Nanjing. I received a warm welcome, and it was quite an interesting and educational experience. To the members of my examination committee, thank you for thoroughly reading my thesis, even though it turned out to be a bit longer than expected. Also a special shout-out to Clément, your help in getting the amrvac-yt frontend project on track was invaluable. Our (many) discussions over those six months were quite fruitful, and I know for sure that without your help the project would never have seen the light of day.

Of course my childhood friends also deserve a huge word of thanks. Dries, Kasper, Joren, Joeri, Jelle; and all my other friends, you were always there to remind me that science is not the only important thing in life. Thanks for the many movie nights, boardgame days and all other fun activities we did. I really needed those to get my mind off things when I was struggling with the science, though you were always ready to listen when I excitedly talked about my ongoing research.

Last but not least, my parents and sister. Where should I begin? There are no words to describe my gratitude, you were always there to support me, through good times and not-so-good times, hearing me rant about my research. I wouldn't have gotten through these four years without your help, and especially your support during the work-from-home year(s) was invaluable (including the endless supply of tea and M\&M's that magically appeared on my desk at home). Thank you for being there for me, again and again.
\begin{flushright}
  \emph{Niels}
\end{flushright}



\cleardoublepage
