\chapter{Outlook} \label{ch: conclusions}

\begin{chapterquote}[J.R.R. Tolkien][The Hobbit, or There and Back Again][Chapter 4]
  There is nothing like looking, if you want to find something... You certainly usually find something,
  if you look, but it's not always quite the something you were after.
\end{chapterquote}

\subsubsection{The road behind}
In this thesis we discussed new and exciting ways to look at linear wave modes and stability of both fluids and plasmas.
We started out with some general background on MHD spectroscopy, and what it physically means for a given configuration to be unstable. Special attention was given to non-adiabatic contributions to the MHD equations, and as extensively discussed in Chapter \ref{ch: spectroscopy} it was shown that including these terms in our formalism causes up-down symmetry breaking in the spectrum: wave modes may shift from purely propagating, stable waves to either damped or overstable modes, depending on their location in the complex plane. Perhaps the most important consequence of considering non-adiabatic effects is the thermal mode, which shifts away from its ideal, zero solution to an either purely unstable or purely damped counterpart. This in turn has a major physical effect on equilibrium plasmas, since it introduces thermal instability and the possibility for the system to form high density, low temperature regions. We dedicated the entirety of Chapter \ref{ch: thermal instability} to the discussion of thermal instabilities, starting from a spectral viewpoint in homogeneous configurations and continued with follow-up studies using fully nonlinear simulations. This yielded some new insights in the evolution of thermal instability, in which we could trace back nonlinear fragmentation of high density filaments to ram pressure imbalances earlier in the formation process. As we mainly relied on a spectroscopic foundation, that is, looking at unstable linear wave modes and using those to trigger thermal instability, we were strongly limited to homogeneous backgrounds in order to make an analytical approach possible.

The homogeneous treatment of equilibrium plasmas early on essentially served as a stepping stone for a more ambitious project: the development of a completely novel tool to solve the linear MHD equations, thereby accounting for plasma inhomogeneity and various physical effects in the formalism. This allows for spectroscopy of realistic plasma configurations, a field in which some early research has been done one to two decades ago, but still has not received the recognition it actually deserves. We named this new tool {\legolas} and extensively discussed its implementation details in Chapter \ref{ch: legolas}, following up with a plethora of different setups from existing literature to thoroughly test and validate the new code in Chapter \ref{ch: legolas_applications}. Finally, as this thesis is mainly targeted towards thermal instabilities from a spectroscopic viewpoint, we applied {\legolas} to a realistic solar atmosphere model in Chapter \ref{ch: solar_atmosphere}; and for the first time ever did a detailed spectroscopic analysis of both the lower solar corona and solar chromosphere. We illustrated clearly that thermal instabilities and magnetothermal modes -- both discrete ones and continuum modes -- are ubiquitous throughout the entire solar atmosphere. The spectral insights gained here hint that the implications of this should not be underestimated: the sheer complexity of the spectra, even with the relatively simple magnetic field configurations used, indicate that the actual quantification of eigenmodes in these kind of setups is far from straightforward; and may yield new insights in both the stability and behaviour of various wave modes present in the solar atmosphere.

As is common practice in a discussion, we could reiterate the results obtained throughout the Chapters in this thesis and highlight the key points. We will not do that here. To be fair, we already provided a detailed discussion near the end of each Chapter, and their respective introductory parts situate each Chapter's contents in a broader scientific context. Especially Chapters \ref{ch: legolas} and \ref{ch: legolas_applications} highlight the extreme importance of (M)HD spectroscopy and the numerous benefits gained by performing in-depth spectroscopic studies, so we will not discuss those (again) here. Instead, we will focus on the road that follows. The development of {\legolas} can be marked as the crowning achievement of this four-year PhD, and now we have a new, modern, and versatile numerical tool to tackle the linear MHD equations with. The versatility, modularity and extensiveness of the {\legolas} framework allows us to investigate plasma and fluid configurations that have never been investigated before, and has opened a new door into the fascinating world of (M)HD spectroscopy. Naturally, the question that follows is obviously ``\emph{what's next?}''.

\subsubsection{The road ahead}
The question ``what's next'' does not have a definitive answer, due to the simple reason that the possibilities are endless. From purely a coding standpoint, this implies improvements, optimisations and extensions. Especially in the corner of optimisations there is a lot of headroom to be gained, both in terms of memory management and CPU usage. For the moment {\legolas} has multiple ways implemented to solve the eigenvalue problem: the first one is using the QR algorithm, which essentially means writing the eigenvalue problem in the form $\bmat^{-1}\amat\statevec = \omega\statevec$. This relies on an inversion of the $\bmat$-matrix, which is not a problem per se since we always ensure that it is properly conditioned. While this method is a ``quick-and-dirty'' way to solve for all eigenvalues and eigenvectors, it fails to exploit the large-scale tri-diagonal structure of the matrices in the eigenvalue problem. The multiplication with the inverse of $\bmat$ lifts the sparsity of the matrices, and we are essentially solving a dense counterpart of the problem instead. Computationally speaking (much) more ``work'' is being done than is actually necessary, though this is still manageable at resolutions up to a few 1000 gridpoints (at most). While these resolutions are surely sufficient for most practical purposes (cfr. Chapter \ref{ch: legolas_applications}), the QR algorithm is not suited if we want to perform runs at extreme resolutions of (tens of) thousands of gridpoints. As {\legolas} currently stores (most of) the $\amat$ and $\bmat$ matrices in memory, we will hit a memory wall pretty fast at those extreme resolutions as well.

The first stop on the road to improving the code would be updating the currently implemented Arnoldi-based solvers (an iterative procedure which is highly efficient for sparse matrices) and modify them to efficiently exploit the sparsity of the matrices. As solving the matrix eigenvalue problem takes up more than $\sim 95\%$ of the actual runtime, especially for high resolutions, there is a lot of benefit to be gained here. Doing this is also quite straightforward -- as is actually the case for any future modification or extension -- since the testing framework covers most of the code base and will ensure that all previous test cases (both unit and regression tests) stay validated. Additionally, once these optimisations are done and both CPU time and memory are efficiently managed, it will be worth looking into parallelisation of the eigenvalue solvers. Currently {\legolas} runs on a single core (but \textsf{Pylbo} can run multiple instances of {\legolas} in parallel), and if we ever hope to exploit High-Performance Supercomputing platforms parallelisation across a large number of CPUs is an absolute requirement. Complementary to this we can look at GPU-based eigenvalue solvers, although this would be a major undertaking, albeit with major benefits, and would be an entire project in itself.

\subsubsection{Many lanes to merge into}
As the {\legolas} framework is highly modular, extending it with additional physics would ``simply'' be a matter of calculating the linearised terms and the matrix elements, and adding them to the code. This opens the door to numerous extensions: from ambipolar diffusion to multi-fluid MHD, in which the latter essentially means that the size of the matrices changes (since then we have more equations and perturbed variables) but the underlying framework will remain exactly the same -- the selfgravity extension is a perfect example here. Adding these kinds of ``new'' physics will in turn allow for spectroscopic studies of completely uninvestigated linear MHD regimes, and will without a doubt yield novel insights into the stability and behaviour of a myriad of previously unexplored physical configurations.

As a side note, the above also implies that we are actually not \emph{limited} to MHD. With the framework in place, any set of equations can be substituted in, provided the matrix elements are known and they can be written in terms of the weak Galerkin formalism. In that respect we can for example look at a fully incompressible variant of the MHD equations we have now, instead of using incompressibility assumptions based on terms containing $\gamma$. A subset with the purely hydrodynamic equations is also possible, where the induction equation and all magnetic field terms are omitted from the formalism, resulting in reduced complexity and smaller matrices in the eigenvalue problem.

Currently ongoing improvements and extensions are related to what we did in Chapter \ref{ch: thermal instability}: perform nonlinear simulations based on the spectrum, thereby exciting specific wave modes and investigate their initial linear, and eventual nonlinear temporal evolution. With that goal in mind modifications are currently being developed that will allow direct linking between {\legolas} and MPI-AMRVAC; given a certain equilibrium, calculate its full spectrum with {\legolas}, setup that particular equilibrium in MPI-AMRVAC, select a given wave mode, impose its eigenfunctions on the background equilibrium state and let the system evolve in time. This is quite an ambitious project, in the sense that it will be an attempt to directly link nonlinear evolutions back to linear theory, and may give novel indications of for example fine structure possibly being a direct result of spectral features.


\subsubsection{An endless number of on- and off-ramps}
Up to now we discussed a couple of ideas for possible improvements and future developments of {\legolas}, but it is also worth elaborating on various \emph{physical} applications. An immediate idea is the combination of non-adiabatic effects and background flows, since {\legolas} is the first linear MHD code that incorporates both effects to such an extent, implying that these kind of configurations have never been explored in earnest. From observations we know that for example the solar atmosphere is full of features where both effects are ubiquitously present, and hence we can immediately start looking at various equilibrium configurations that are important in solar atmospheric applications, shining new light on the complexities and intricacies of various instabilities present therein. This can in turn deepen our understanding of e.g. how and why prominences and corresponding fine structure forms, but will also be relevant in various other astrophysical applications: from accretion- and proto-planetary disks, to jets around black holes, solar wind settings, etc.

One of the exciting new prospects is a linear approach to the study of turbulence, which is inherently a highly nonlinear process in itself. Background flows are known to be an essential ingredient for the development of turbulent behaviour in fluids and plasmas, and in that respect we can start looking at this from a spectral viewpoint. Although this seems to be an apparent contradiction in terms, we can ask whether the onset of turbulence can be traced back to spectral features. In other words, if the entire spectrum of a given state is calculated at some point during its dynamical evolution, it may be possible that clear changes present themselves within the spectrum at exactly the parameter values corresponding to the nonlinear features. Continuing this train of thought, how important is or are the most unstable modes of a given system? Do these play a dominant role, governing the behaviour of the plasma over a given time scale? These questions can only be answered by an in-depth spectral study of selected configurations, complemented with high-resolution fully nonlinear simulations. In that respect we bridge the gap between direct numerical simulations and spectral theory, highlighting the importance and applicability of codes like {\legolas} in other disciplines as well.

We started our journey on the exciting road that is spectral theory, gained new knowledge, developed new tools, and expanded the view around us. We can now really start looking forward, where numerous off-ramps are coming in sight, allowing us to branch off the main road and start applying our new spectroscopic toolbox to a multitude of physical disciplines. A myriad of opportunities and ideas will naturally present themselves as we are improving and applying {\legolas}: the possibilities are endless, both in terms of applications and extensions, and research in this field will without a doubt yield novel insights in numerous linear instability studies in the years to come.



% discuss applications with respect to instabilities, maybe mention stuff put into the numerous application ideas (linear-nonlinear, turbulence, new modes, space weather applications, etc.)


\cleardoublepage
