% All shortcuts and symbols are to be defined here

% radiative cooling
\newcommand{\HLF}{\mathscr{L}}
\nomenclature{$\HLF$}{the heat-loss function, defined as energy losses minus energy gains}
\newcommand{\HLFcool}{\Lambda(T)}
\nomenclature{$\HLFcool$}{the cooling curve}
\newcommand{\HLFheat}{\mathcal{H}}
\nomenclature{$\HLFheat$}{the heating function}
\newcommand{\dHLFT}{\HLF_\text{T}}
\nomenclature{$\dHLFT$}{temperature derivative of the heat-loss function}
\newcommand{\dHLFrho}{\HLF_\rho}
\nomenclature{$\dHLFrho$}{density derivative of the heat-loss function}

% thermal conduction
\newcommand{\bkappa}{\boldsymbol{\kappa}}
\nomenclature{$\bkappa$}{the thermal conductivity tensor}
\newcommand{\idmat}{\boldsymbol{I}}
\nomenclature{$\idmat$}{the identity matrix}
\newcommand{\kappapara}{\kappa_\parallel}
\nomenclature{$\kappapara$}{the thermal conduction coefficient parallel to the magnetic field lines}
\newcommand{\kappaperp}{\kappa_\perp}
\nomenclature{$\kappaperp$}{the thermal conduction coefficient perpendicular to the magnetic field lines}
\newcommand{\skappapara}[1]{\kappa_{\parallel #1}}
\newcommand{\skappaperp}[1]{\kappa_{\perp #1}}
\newcommand{\kappazeropf}{\frac{\skappapara{0} - \skappaperp{0}}{B_0^2}}
\newcommand{\kappaonepf}{\frac{\skappapara{1} - \skappaperp{1}}{B_0^2}}
\newcommand{\kappapfK}[1]{\text{K}_{#1}}

% vector quantities
\newcommand{\unit}[1]{\hat{\boldsymbol{e}}_{\text{#1}}}
\newcommand{\munit}[1]{\hat{\boldsymbol{e}}_{#1}}
\newcommand{\uunit}[1]{\hat{\boldsymbol{u}}_{#1}}
\nomenclature{$\unit{B}$}{a unit vector along the magnetic field}
\newcommand{\bv}{\boldsymbol{v}}
\nomenclature{$\bv$}{velocity vector}
\newcommand{\bb}{\boldsymbol{B}}
\nomenclature{$\bb$}{magnetic field vector}
\newcommand{\bx}{\boldsymbol{x}}
\nomenclature{$\bx$}{position vector}
\newcommand{\ba}{\boldsymbol{A}}
\nomenclature{$\ba$}{magnetic vector potential such that $\bb = \nabla \times \ba$}
\newcommand{\bk}{\boldsymbol{k}}
\nomenclature{$\bk$}{wave vector}
\newcommand{\kpara}{k_\parallel}
\nomenclature{$\kpara$}{component of the wave vector $\bk$ parallel to $\bb$}
\newcommand{\kperp}{k_\perp}
\nomenclature{$\kperp$}{component of the wave vector $\bk$ perpendicular to $\bb$}
\newcommand{\bg}{\boldsymbol{g}}
\nomenclature{$\bg$}{gravitational field vector}
\newcommand{\bu}{\boldsymbol{u}}

% matrices
\newcommand{\statevec}{\boldsymbol{X}}
\nomenclature{$\statevec$}{notation for the state vector in an eigenvalue problem}
\newcommand{\amat}{\mathcal{A}}
\nomenclature{$\amat$}{the A-matrix in a standard eigenvalue problem $\amat\statevec = \omega\statevec$}
\newcommand{\bmat}{\mathcal{B}}
\nomenclature{$\bmat$}{the B-matrix in a general eigenvalue problem $\amat\statevec = \omega\bmat\statevec$}

% shortcuts
\newcommand{\icomplex}{\text{i}}
\nomenclature{$\icomplex$}{the complex number $\icomplex$ such that $\icomplex^2 = -1$}
\newcommand{\gmone}{(\gamma - 1)}
\nomenclature{$\gamma$}{ratio of specific heats}
\newcommand{\alfvenspeed}{c_\text{A}}
\nomenclature{$\alfvenspeed$}{Alfv\'en speed}
\newcommand{\soundspeed}{c_\text{s}}
\nomenclature{$\soundspeed$}{sound speed}
\newcommand{\isosoundspeed}{c_\text{i}}
\nomenclature{$\isosoundspeed$}{isothermal sound speed}
\newcommand{\kcrit}{k_\text{crit}}
\nomenclature{$\kcrit$}{the critical wave number above which thermal conduction stabilises TI}
\newcommand{\halpha}{\text{H}\alpha}
\nomenclature{$\halpha$}{hydrogen-alpha}
\newcommand{\eps}{\varepsilon}
\nomenclature{$\eps$}{the scale factor used to switch between Cartesian and cylindrical geometries}
\newcommand{\epsinv}{\varepsilon^{-1}}
\newcommand{\magneticreynolds}{R_\text{m}}
\nomenclature{$\magneticreynolds$}{the magnetic Reynolds number}


% constants
\newcommand{\massp}{m_\text{p}}
\nomenclature{$\massp$}{the proton mass}
\newcommand{\masse}{m_\text{e}}
\nomenclature{$\masse$}{the electron mass}
\nomenclature{$c$}{the speed of light}
\newcommand{\boltzmanncte}{k_\text{B}}
\nomenclature{$\boltzmanncte$}{the Boltzmann constant}
\newcommand{\electroncharge}{e}
\nomenclature{$\electroncharge$}{the electron charge}

% definitions
\newcommand{\Qi}{\mathcal{Q}_\text{i}}
\newcommand{\Qifull}{\rho_0^2\dHLFrho - \left(\kappapara \kpara^2 + \rho_0 \dHLFT\right)\isosoundspeed^2}
\newcommand{\Qai}{\mathcal{Q}_\text{ai}}
\newcommand{\Qaifull}{
  \rho_0^2\dHLFrho - \left(\kappapara \kpara^2 + \rho_0 \dHLFT\right)\left(\alfvenspeed^2 + \isosoundspeed^2\right)
}


% text shortcuts
\newcommand{\jccorona}{\textsf{JC corona}}
\newcommand{\spexdm}{\textsf{Spex DM}}
\newcommand{\legolas}{\textsf{Legolas}}
