\chapter*{Beknopte samenvatting} \label{ch: abstract_nl}
\addcontentsline{toc}{chapter}{\nameref{ch: abstract_nl}}
In deze thesis onderzoeken we een magnetohydrodynamische spectrale aanpak tot golven en instabiliteiten, wat ons een nieuwe kijk geeft op de verschillende golven aanwezig in vloeistoffen en plasmas en ons toelaat om instabiel gedrag te bestuderen. We kijken in het bijzonder naar thermische instabiliteiten, waarvan we weten dat deze verantwoordelijk zijn voor condensaties met hoge dichtheid en lage temperatuur in de corona van de zon; bijgevolg zijn ze het meest waarschijnlijke mechanisme voor de vorming van coronale regen en prominensen. Deze thesis is opgedeeld in twee grote delen: eerst kijken we naar de spontane ontstaan en de evolutie van fijnstructuur in regio's met hoge dichtheid, gevormd door thermische instabiliteiten in een evenwicht vergelijkbaar met de corona van de zon. We doen dit met behulp van een combinatie van analytische methoden en numerieke simulaties met hoge resolutie in twee en drie dimensies. We beschouwen een spectrale aanpak, dat wil zeggen, we baseren ons op de groeiritmes en stabiliteit van thermische en trage MHD golven, en gebruiken hun eigenfuncties om thermisch instabiel gedrag te veroorzaken. Ten tweede ontwikkelen we een volledig nieuwe code, {\legolas}, met als doel om spectrale studies te doen van inhomogene plasmas. Deze code kunnen we vervolgens gebruiken om eigenspectrum studies te doen van een realistisch, empirisch model van de zonne-atmosfeer, waarbij we alle eigenwaarden, eigenfuncties, en stabiliteitsregio's kunnen kwantificeren.

We beginnen met een vrij kort inleidend Hoofdstuk, waarbij we een algemene achtergrond van de zonne-atmosfeer geven samen met de belangrijke rol van instabiliteiten in het dynamisch gedrag ervan. We houden dit hoofdstuk vrij kort met opzet, aangezien we elk individueel hoofdstuk beginnen met een uitgebreide inleiding waarbij het overeenstemmend onderzoek in een bredere context geplaatst wordt. In Hoofdstuk \ref{ch: spectroscopy} kijken we naar de basisprincipes van MHD spectroscopie, waarbij we een algemene introductie geven van zowel de ideale als niet-adiabatische MHD vergelijkingen. We bespreken de typische aanpak voor de linearisatie rond een bepaalde achtergrond en het berekenen van de eigenwaarden, dus het spectrum. We gaan over spectrale symmetrieën, en bekijken een algemene benadering van inhomogene plasmas op basis van meerdere homogene lagen. We sluiten dit Hoofdstuk af met een kijk op het ontstaan van de MHD continua, welke van groot belang zijn in spectrale theorie.

In Hoofdstuk \ref{ch: thermal instability} gebruiken we de theorie van het vorige Hoofdstuk om de eigenfuncties van trage MHD golven voor een bepaalde configuratie te kwantificeren. Deze gaan we gebruiken als perturbaties in numerieke simulaties, waarbij we trage MHD golven aanslaan onder zonne-coronale condities. De gedempte trage golven interageren met elkaar in een regime dat onstabiel is voor thermische modes, welke uiteindelijk thermische instabiliteiten gaan veroorzaken en condensaties met hoge dichtheid gaat vormen. We volgen deze condensaties vervolgens diep in het niet-lineaire regime. We bespreken het ontstaan van zogenaamde rebound shocks bij de vorming van de filamenten, en hoe verschillen in ram druk uiteindelijk leiden tot fragmentatie van de hoge-dichtheidsfilamenten in kleinere, dichte blobs. We zien dat fijnstructuur natuurlijk ontstaat en bespreken de link tussen de oriëntatie van deze structuur ten opzichte van het magneetveld, waarbij we gebruik maken van synthetische $\halpha$ visualisaties.

De wiskundige achtergrond van de {\legolas} code wordt besproken in Hoofdstuk \ref{ch: legolas}, waarbij we in detail het ganse formalisme bekijken. De niet-ideale MHD vergelijkingen worden uitgebreid met verschillende fysische effecten, vervolgens gaan we deze lineariseren rond een drie-dimensionale achtergrond met één-dimensionale variatie, gevolgd door een Fourier-analyse. De bekomen vergelijkingen worden gediscretiseerd met behulp van Eindige Elementen in de belangrijke hoogte of radiale coördinaat, waarbij we zowel Cartesiaanse als cilindrische geometrieën beschouwen. Het zwak Galerkin-formalisme wordt gebruikt om de vergelijkingen te transformeren naar een algemeen, complex eigenwaarde probleem, en we bespreken in detail hoe de matrices samengesteld worden vertrekkende vanuit de vergelijkingen en hoe de randvoorwaarden opgelegd worden. In Hoofdstuk \ref{ch: legolas_applications} testen we de nieuwe code met behulp van gekende resultaten, gebaseerd op vroeger berekende spectra uit de literatuur of analytisch bekomen dispersierelaties. We bekijken $p$ en $g$ modes in een magnetische en gestratificeerde atmosfeer, modes relevant voor coronale seismologie, thermische instabiliteiten, Kelvin-Helmholtz instabiliteiten, resistieve tearing modes, magneto-thermische instabiliteiten, en nog vele andere. De hoge resoluties die gebruikt worden in deze gevallen geven ons nieuwe inzichten in vroeger berekende spectra.

In Hoofdstuk \ref{ch: solar_atmosphere} tenslotte doen we voor de eerste keer ooit een gedetailleerde, hoge-resolutie spectrale studie van een volledig realistisch zonne-atmosferisch model met behulp van {\legolas}, en berekenen het volledige spectrum met eigenfuncties. We brengen gravitationale effecten in rekening, samen met optisch dunne stralingsverliezen en thermische conductie, waarbij we speciale aandacht geven aan thermische instabiliteiten en het gedrag van de trage en thermische continua in verschillende coronale en chromosferische settings. We treffen verschillende regio's aan waar de thermische, trage, en snelle modes allemaal instabiele oplossingen hebben, samen met interessant gedrag van de trage en thermische continua die samensmelten op de imaginaire spectrale as. De algemene conclusie van dit hoofdstuk is dat instabiele thermische en magneto-thermische oplossingen alomtegenwoordig zijn in de atmosfeer van de zon, en zijn waarschijnlijk verantwoordelijk voor het merendeel van de fijnstructuur die we observeren.

We sluiten deze thesis af met een conclusie, waarbij we ons niet toeleggen op het herhalen van de resultaten van vorige Hoofdstukken, maar expliciet vooruitkijken naar de implicaties van het onderzoek, mogelijke extensies en toekomstige onderzoekstopics. Met de ontwikkeling van {\legolas} hebben we een nieuwe mijlpaal bereikt voor spectrale studies van zowel vloeistoffen als plasmas, wat ons zal toelaten om MHD configuraties te bestuderen die nog nooit onderzocht zijn, en waardoor we de nog vele geheimen van MHD spectroscopie kunnen beginnen ontrafelen.


\cleardoublepage
