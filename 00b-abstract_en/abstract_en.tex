\chapter*{Abstract} \label{ch: abstract}

This thesis centres itself around a magnetohydrodynamic spectroscopy approach to waves and instabilities, which provides a fresh way to look at various wave modes at play in fluids or plasmas and allows us to study unstable behaviour. We pay special attention to thermal instabilities, which are known to be responsible for high-density, low-temperature condensations in the solar corona and are the most probable mechanism for the formation of coronal rain and solar prominences. The main road taken by this thesis is split into two parts: first, we look at the spontaneous emergence and evolution of filamentary fine structure in high-density regions, which are formed through the process of thermal instability under solar coronal conditions. This is done through a combination of analytical methods, combined with high-resolution simulations in two and three dimensions, where a spectroscopic basis; that is, the growth rates and stability of both thermal and slow MHD wave modes along with their eigenfunctions, is used to trigger thermally unstable behaviour. Second, we develop a completely novel tool named {\legolas} to perform eigenspectrum studies of inhomogeneous plasmas, which we then apply to a realistic empirical solar atmosphere model, allowing us to quantify all eigenvalues and eigenfunctions, while also probing stability regions.

We start with a rather short introductory Chapter, where we provide some general background on the solar atmosphere and the important role instabilities play in its dynamics. This was kept intentionally brief, as the individual Chapters themselves will have a dedicated introduction, situating the research in a broader context. In Chapter \ref{ch: spectroscopy} we go over the main principles of MHD spectroscopy, providing a general introduction to both the ideal and non-adiabatic MHD equations and how one usually goes about linearisation around a given background and calculate the corresponding eigenvalues, and hence the spectrum. We discuss spectral symmetry, and look at a rough approximation to inhomogeneous plasmas based on homogeneous layers. Finally, we close with a fresh look on the emergence of MHD continua, which are of paramount importance in spectral theory.

In Chapter \ref{ch: thermal instability} we use the theory from the previous Chapter to quantify slow wave eigenfunctions for a given configuration, and use those as perturbations to excite slow MHD waves in numerical simulations under solar coronal conditions. The (damped) slow waves are interacting in a regime that is unstable to the thermal mode, eventually triggering thermal instability and forming high-density regions which we trace far into the nonlinear regime. We discuss the emergence of rebound shocks when the filament is forming, and how ram pressure differences eventually lead to fragmentation of the high-density filament into smaller, dense blobs. Thread-like features emerge naturally and we discuss the link between fine structure orientation and magnetic field alignment, supplemented with synthetic {$\halpha$} views.

The mathematical foundations behind the {\legolas} code are discussed in Chapter \ref{ch: legolas}, where we provide a detailed overview of the formalism employed. The non-ideal MHD equations, including a plethora of additional physics, are linearised around a three-dimensional background with one-dimensional variation and Fourier-analysed in the ignorable coordinates. The resulting set of linearised equations is discretised using a Finite Element representation in the important height or radial variation, handling both Cartesian and cylindrical geometries. The weak Galerkin formalism is used to transform the equations into a generalised complex eigenvalue problem, and we go in-depth on how the matrices are assembled from the set of equations and how the relevant boundary conditions are applied. In Chapter \ref{ch: legolas_applications} we test our new code on a plethora of well-established results, based on previously calculated spectra in literature or analytically derived dispersion relations. We showcase $p$ and $g$ modes in magnetised, stratified atmospheres, modes relevant for coronal loop seismology, thermal instabilities, Kelvin-Helmholtz instabilities, resistive tearing modes, magnetothermal instabilities, and many more. The high resolutions used in all cases shed new light on previously calculated spectra.

Finally, in Chapter \ref{ch: solar_atmosphere}, we perform for the first time ever a detailed, high-resolution eigenspectrum study of a fully realistic solar atmosphere model {\legolas}, thereby calculating the full spectrum with corresponding eigenfunctions. We include external gravity, optically thin radiative losses, and thermal conduction, with special attention given to thermal instabilities along with a fresh outlook on the behaviour of the slow and thermal continua in different coronal and chromospheric regions. We encounter whole regions where the thermal, fast, and slow modes all have unstable solutions, along with interesting behaviour of the thermal and slow continua, which can merge on the imaginary spectral axis. Main conclusions drawn here discuss that thermal instabilities and propagating overstable magnetothermal modes are ubiquitously present throughout the solar atmosphere, and may very well be responsible for much of the observed fine structure.

Instead of concluding this thesis with a recapitulation of obtained results, we instead discuss possible implications and extensions, along with future research possibilities. The development of {\legolas} marks a new milestone in eigenspectrum studies of both fluids and plasmas and will allow us to investigate hitherto unexplored MHD equilibria, such that we can unravel the many secrets of (M)HD spectroscopy to a degree that was not possible before.




\cleardoublepage
